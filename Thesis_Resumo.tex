%%%%%%%%%%%%%%%%%%%%%%%%%%%%%%%%%%%%%%%%%%%%%%%%%%%%%%%%%%%%%%%%%%%%%%%%
%                                                                      %
%     File: Thesis_Resumo.tex                                          %
%     Tex Master: Thesis.tex                                           %
%                                                                      %
%     Author: Andre C. Marta                                           %
%     Last modified :  2 Jul 2015                                      %
%                                                                      %
%%%%%%%%%%%%%%%%%%%%%%%%%%%%%%%%%%%%%%%%%%%%%%%%%%%%%%%%%%%%%%%%%%%%%%%%

\section*{Resumo}

% Add entry in the table of contents as section
\addcontentsline{toc}{section}{Resumo}

Atualmente, as unidades de processamento gráfico (do inglês GPUs) são os principais dispositivos computacionais usados para acelerar aplicações de cariz paralelo. Contudo, esse imenso desempenho acarreta um alto consumo energético. Diversas soluções podem ser adotadas para aumentar a eficiência energética desses dispositivos. Porém, o escalonamento de tensão-frequência (T-F) tem sido a solução que obtém o melhor resultado, permitindo melhorar as métricas de eficiência energética de forma automática e independente do tipo de aplicações a ser executadas. 

As implementações atuais de escalonamento dinâmico de tensão e frequência (do inglês DVFS) em GPUs são unidimensionais, ajustando a frequência dentro dos pares de tensão-frequência padrão. No entanto, este ajuste é insuficiente, pois não garante o par T-F mais adequado à aplicação a ser executada. Esta dissertação apresenta uma nova metodologia para caraterizar o impacto do DVFS não convencional em GPUs, capaz de colmatar o carácter unidimensional das implementações actuais. Para atingir este objetivo, a abordagem proposta define um espaço de parametrização que determina a faixa de tensão tolerável para cada frequência. A mesma foi testada em duas GPUs da AMD com os resultados a mostrarem que estes dispositivos em particular, são ambos capazes de operar em segurança com até menos 20\% do valor padrão de tensão.
Este espaço de parametrização definido é então usado pelo mecanismo de otimização T-F desenvolvido para selecionar automaticamente a configuração de maior eficiência energética. Quando aplicado a aplicações de Aprendizagem Profunda e, especificamente, Redes Neurais Convolucionais, o mecanismo de otimização proposto demonstra ser capaz de melhorar a eficiência energética da GPU em até 44\% sem qualquer deterioração medida da precisão do modelo de rede neural a ser treinado.

\vfill

\textbf{\Large Palavras-chave:} Unidade de Processamento Gráfico, Escalonamento Dinâmico de Tensão e Frequência, Redução da Tensão, Mecanismo de Optimização, Aprendizagem Profunda, Redes Neuronais Profundas.
