%%%%%%%%%%%%%%%%%%%%%%%%%%%%%%%%%%%%%%%%%%%%%%%%%%%%%%%%%%%%%%%%%%%%%%%%
%                                                                      %
%     File: Thesis_Abstract.tex                                        %
%     Tex Master: Thesis.tex                                           %
%                                                                      %
%     Author: Andre C. Marta                                           %
%     Last modified :  2 Jul 2015                                      %
%                                                                      %
%%%%%%%%%%%%%%%%%%%%%%%%%%%%%%%%%%%%%%%%%%%%%%%%%%%%%%%%%%%%%%%%%%%%%%%%

\section*{Abstract}

\addcontentsline{toc}{section}{Abstract}

Nowadays, graphics Processing Units (GPUs) are the primary computational devices used to accelerate highly parallel applications. However, this immense performance comes at the cost of high energy consumption. Several solutions can be adopted to increase the energy-efficiency of these devices. Though, Voltage-Frequency (V-F) scaling has been the one that achieves the best results, by allowing to improve this metric automatically and independently of the workload. 
However, current implementations of Dynamic Voltage and Frequency Scaling (DVFS) on GPUs are still one-dimensional, by simply adjusting frequency while relying on default voltage settings. To overcome this limitation, this dissertation introduces a new methodology to fully characterize the impact of non-conventional DVFS on GPUs. To attain this objective, the proposed approach defines a Usable Execution Space (UES) that determines the tolerable voltage range allowed by each frequency. The conducted experimental evaluation, using two out-of-the-shelf AMD GPUs, demonstrated that these particular devices are able to be safely undervolted by more than 20\%. The devised UES is then used by the conceived V-F optimization mechanism to automatically select the most energy-efficiency configuration. When applied to Deep Learning applications and, specifically, Convolutional Neural Networks (CNNs), the proposed optimization mechanism can improve the GPU energy efficiency by up to 44\% without any measured deterioration of the CNN model accuracy.

\vfill

\textbf{\Large Keywords:} Graphics Processing Unit, Dynamic Voltage and Frequency Scaling, Undervoltage, Optimization Mechanism, Deep Learning, Deep Neural Networks.

